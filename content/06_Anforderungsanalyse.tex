% vim: set tw=160:

\chapter{Anforderungen an ein \protect\glstext*{firefox-addon}}
\index{Anforderungen}
\label{chap:Anforderungen}
Um die im Rahmen dieser Arbeit gewonnen Erkenntnisse zum Einsatz von \gls{xml-dsig} in \gls{xhtml} nutzbar zu machen, wäre es erforderlich einen Prototypen
eines Programms zu entwickeln, der eine einfache Umsetzung der hier vorgestellten Verfahren und Mechanismen demonstriert. Da \gls{xml-dsig} für ein sehr weit
gefasstes Einsatzfeld entwickelt wurde, sollen hier nun einige grundlegende Anforderungen an eine solche Software formuliert werden. Um die Formulierungen zu
vereinfachen, wird davon ausgegangen, dass diese Software als \gls{firefox-addon} entwickelt wird.

\section{Signaturerkennung}
\index{Signaturerkennung}
\label{sec:Anforderungen:Signaturerkennung}
Das AddOn muss beim Laden einer \gls{webseite} in \gls{firefox} ein \gls{xhtml}-Dokument erkennen und es auf darin enthaltene Signaturen nach \gls{xml-dsig}
durchsuchen. Dazu prüft es das Dokument zuerst auf das Vorhandensein eines \gls{xml}-Headers und wertet dann eine oder mehrere der in
\fref{chap:Signaturbindung} beschriebenen Varianten zur Signaturbindung aus.

\section{Signaturauswertung}
\index{Signaturauswertung}
\label{sec:Anforderungen:Signaturauswertung}
In \gls{xhtml}-Dokumenten gefundene \gls{xml-dsig}-Signaturen müssen ausgewertet werden, wofür zwei Abläufe, "`Reference Validation"' und "`Signature
Validation"' einzuhalten sind.

\subsection{Reference Validation}
Der Inhalt von \xmlelem{SignedInfo} aus der Signatur wird mittels \gls{c14n} in eine kanonische Form gebracht. Danach wird jede \xmlelem{Reference} aufgelöst
und die referenzierten Daten geladen. Diese Daten werden in der Reihenfolge ihrer Definition durch die Transformationen aus \xmlelem{Transform} geleitet und
umgewandelt. Die Daten aus der letzten Transformation werden in den Algorithmus, der in \xmlelem{DigestMethod} angegeben ist, geleitet, wodurch eine
Hash-Prüfsumme erzeugt wird. Diese wird mit dem Wert aus \xmlelem{DigestValue} verglichen. Sind die beiden Werte nicht identisch, schlägt die Validierung der
Signatur fehl. Nachdem dieser Vorgang für jede \xmlelem{Reference} durchgeführt wurde, beginnt die "`Signature Validation"'.

\subsection{Signature Validation}
Das in \xmlelem{KeyInfo} spezifizierte \gls{openpgp}-Zertifikat wird aus dem lokalen Speicher, oder, falls dort nicht vorhanden, von einem Schlüsselserver
geladen. Die Signaturmethode muss über die kanonische Form von \xmlelem{SignatureMethod} ermittelt werden und wird dann zusammen mit dem Zertifikat und der
kanonischen Form von \xmlelem{SignedInfo} zur Verifizierung der Gültigkeit der Signatur in \xmlelem{SignatureValue} benutzt. Ist die Verifizierung erfolgreich,
wird die Signatur vom \gls{firefox-addon} als gültig anerkannt und das Ergebnis sollte grafisch innerhalb des Darstellungsbereichs von \gls{firefox} angezeigt
werden.

\section{Abstrakte Schnittstelle für Signaturprüfung}
\label{sec:Anforderungen:Schnittstelle}
Um \gls{openpgp} und \gls{x509} gleichzeitig unterstützen zu können, sollten Abläufe, die von der Art der Signatur abhängig sind, durch eine
Programmierschnittstelle abstrahiert werden. Das AddOn sollte damit in der Lage sein für beide Verfahren die selbe äußere Darstellung und das selbe Verhalten
während der Prüfung von Signaturen anbieten zu können
