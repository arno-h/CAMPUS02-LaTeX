% vim: set tw=160:

\chapter{Anforderungskatalog für ein \protect\glstext*{firefox-addon}}
\index{Anforderungskatalog}
\label{chap:Anforderungskatalog}

\section{Signaturerkennung}
\index{Signaturerkennung}
\label{sec:Anforderungskatalog:Signaturerkennung}
Das AddOn muss beim Laden einer \gls{webseite} ein \gls{xhtml}-Dokument erkennen und es auf darin enthaltene Signaturen nach \gls{xml-dsig}
durchsuchen. Dazu prüft es das Dokument zuerst auf das Vorhandensein eines \gls{xml}-Headers und wertet dann die \xmlelem{link}-Elemente mit dem Wert
\texttt{xml-dsig} im Attribut \xmlattr{link}{rel} aus.

\section{Signaturauswertung}
\index{Signaturauswertung}
\label{sec:Anforderungskatalog:Signaturauswertung}
In \gls{xhtml}-Dokumenten gefundene \gls{xml-dsig}-Signaturen müssen ausgewertet werden, dafür müssen zwei Abläufe, "`Reference Validation"' und "`Signature
Validation"' eingehalten werden.

\subsection{Reference Validation}
Der Inhalt von \xmlelem{SignedInfo} aus der Signatur wird mittels \gls{c14n} in eine kanonische Form gebracht. Danach wird jede \xmlelem{Reference} aufgelöst
und die referenzierten Daten geladen. Diese Daten werden in der Reihenfolge ihrer Definition durch die Transformationen aus \xmlelem{Transform} geleitet und
umgewandelt. Die Daten aus der letzten Transformation werden in den Algorithmus, der in \xmlelem{DigestMethod} angegeben ist, geleitet, wodurch eine
Hash-Prüfsumme erzeugt wird. Diese wird mit dem Wert aus \xmlelem{DigestValue} verglichen. sind die beiden Werte nicht identisch, schlägt die Validierung der
Signatur fehl. Nachdem dieser Vorgang für jede \xmlelem{Reference} durchgeführt wurde, beginnt die "`Signature Validation"'.

\subsection{Signature Validation}
Das in \xmlelem{KeyInfo} spezifizierte \gls{openpgp}-Zertifikat wird aus dem lokalen Speicher, oder, falls dort nicht vorhanden, von einem Schlüsselserver
geladen. Die Signaturmethode muss über die kanonische Form von \xmlelem{SignatureMethod} ermittelt werden und wird dann zusammen mit dem Zertifikat und der
kanonischen Form von \xmlelem{SignedInfo} zur Verifizierung der Gültigkeit der Signatur in \xmlelem{SignatureValue} benutzt. Ist die Verifizierung erfolgreich,
wird die Signatur vom \gls{firefox-addon} als gültig anerkannt.

\section{\protect\glstext*{gpg}-Anbindung}
\index{\protect\glstext*{gpg}-Anbindung}
\label{sec:Anforderungskatalog:GnuPG-Anbindung}

