% vim: set tw=160:

\chapter{Grundlagen und Definitionen}
\label{chap:GrundlagenDefinitionen}
\blindtext

Eine Referenz auf die \fref{fig:wot} oder auf die \fref{fig:ht} ist nicht schwer und nimmt, falls nötig, sogar die Seitenzahl mit ein den Text auf. Wer Mut und
Laune hat, kann sich auch mit TIKZ\footnote{\url{http://www.texample.net/tikz/}} beschäftigen um schönere Grafiken als mit Bitmaps zu erhalten.

\begin{figure}
\centering
\begin{tikzpicture}[node distance   = 4 cm]
    \tikzset{Key/.style = {shape          = rectangle,
                           rounded corners,
                           drop shadow,
                           fill           = orange,
                           draw           = black,
                           thick,
                           text           = black,
                           text centered,
                           inner sep      = 3pt,
                           outer sep      = 0pt,
                           minimum size   = 24 pt}}
    \tikzset{DirectTrust/.style   = {thick,
                                     double          = green,
                                     double distance = 1pt}}
    \tikzset{IndirectTrust/.style   = {thin,
                                       double          = blue,
                                       double distance = 1pt}}
    \tikzset{Label/.style =   {draw,
                               fill           = white,
                               text           = black}}
    \node[Key](A){Alice};
    \node[Key,right=of A](B){Bob};
    \node[Key,right=of B](C){Carmen};
    \node[Key,above=of B](D){Dorian};     
    \draw[DirectTrust](B) to node[Label]{vertraut} (D) ;
    \tikzset{DirectTrust/.append style = {bend left}}
    \draw[DirectTrust](A) to node[Label]{vertraut} (B);
    \draw[DirectTrust](B) to node[Label]{vertraut} (C);
    \draw[DirectTrust](D) to node[Label]{vertraut} (C);
    \tikzset{IndirectTrust/.append style = {bend left}}
    \draw[IndirectTrust](A) to node[Label]{vertraut indirekt} (D) ;
    \tikzset{IndirectTrust/.append style = {bend right}}
    \draw[IndirectTrust](A) to node[Label]{vertraut indirekt} (C) ;

\end{tikzpicture}

\caption{Schematische Darstellung des \glsentrydesc{wot}}
\label{fig:wot}
\end{figure}

\begin{figure}
\centering
\begin{tikzpicture}[node distance   = 3 cm]
    \tikzset{Key/.style = {shape          = rectangle,
                           rounded corners,
                           drop shadow,
                           fill           = white,
                           draw           = black,
                           thick,
                           text           = black,
                           text centered,
                           inner sep      = 3pt,
                           outer sep      = 0pt,
                           minimum size   = 24 pt}}
    \tikzset{DirectTrust/.style   = {thick,
                                     double          = black,
                                     double distance = 1pt}}
    \tikzset{IndirectTrust/.style   = {thin,
                                       double          = white,
                                       double distance = 1pt}}
    \tikzset{Label/.style =   {draw,
                               fill           = white,
                               text           = black,
                               text centered,
                               text width     = 4em}}
    \node[Key](CA1){CA1 (Root)};
    \node[below=of CA1](AUX1){};
    \node[Key,left=of AUX1](CA1-1){CA1.1};
    \node[Key,right=of AUX1](CA1-2){CA1.2};
    \node[below=of CA1-1](AUX1-1){};
    \node[Key,left=of AUX1-1](Alice){Alice};
    \node[Key,right=of AUX1-1](Bob){Bob};
    \node[Key,below=of CA1-1](Carmen){Carmen};
    \node[Key,below=of CA1-2](Dorian){Dorian};
    \draw[DirectTrust,bend left](CA1-1) to node[Label]{vertraut} (CA1) ;
    \draw[DirectTrust,bend right](CA1-2) to node[Label]{vertraut} (CA1) ;
    \draw[DirectTrust,bend left](Alice) to node[Label]{vertraut} (CA1-1) ;
    \draw[DirectTrust,bend right](Bob) to node[Label]{vertraut} (CA1-1) ;
    \draw[DirectTrust](Carmen) to node[Label]{vertraut} (CA1-1) ;
    \draw[DirectTrust](Dorian) to node[Label]{vertraut} (CA1-2) ;
    \draw[IndirectTrust,bend right](Carmen) to node[Label]{vertraut indirekt} (Dorian) ;
    \draw[IndirectTrust,bend right](Alice) to node[Label]{vertraut indirekt} (Carmen) ;

\end{tikzpicture}

\caption{Schematische Darstellung eines \glsentrydesc{ht}}
\label{fig:ht}
\end{figure}

Es kann auch aus dem \fref{lst:link} auf die Zeile \ref{lin:link-href} verwiesen werden, ohne dies fix im Text zu inkludieren. Alles wird automatisch über die
Position der Labels berechnet.

\lstinputlisting[language=HTML,
                 caption={Verwendung von \xmlelem{link}},
                 label=lst:link,
                 emph={link,href}]{source/link.html}

Auch Tabellen wie die \fref{tab:xml-namespaces} sind schnell erstellt und sehr gut lesbar. Vor allem sollte auf vertikale Trennlinien verzichtet werden, um die
Lesbarkeit zu erhöhen \cite{latex}.

\begin{table}
    \centering
    \begin{tabularx}{\textwidth}{ l X }
        \gls{xml}-Dialekt  & \gls{xml-ns} \\
        \hline
        \hline
        \gls{xhtml} & \url{http://www.w3.org/1999/xhtml} \\
        \hline
        \gls{xml-dsig} & \url{http://www.w3.org/2000/09/xmldsig\#} \\
        \hline
        \gls{xinclude} & \url{http://www.w3.org/2001/XInclude} \\
        \hline
        \gls{xslt} & \url{http://www.w3.org/1999/XSL/Transform} \\
        \hline
    \end{tabularx}
    \caption{Namespaces für unterschiedliche \protect\gls{xml}-Dialekte}
    \label{tab:xml-namespaces}
\end{table}

