% vim:
%%%------------------------------------
\chapter{Theoretische Grundlagen}
\index{Theoretische Grundlagen}
\label{chp:TheoretischeGrundlagen}

\section{Definitionen}

\subsection{Digitale Signaturen}
Unter dem kryptographischen Sammelbegriff der digitalen Signatur versteht man eine zuverlässige und nicht fälschbare Kontrollinformation die über den Urheber
und mögliche Modifikationen des Inhalts der zu schützenden Daten Auskunft gibt\cite{itgk:bsi}. Dabei kann eine digitale Signatur analog zu der Unterschrift einer Person betrachtet werden\cite{kits}. 

%%%------------------------------------

\section{Digitale Signaturverfahren}
\index{Digitale Signaturverfahren}
\label{sec:thg:dsv}


\subsection{PGP}
\index{\gls{pgp}}
\label{sec:thg:dsv:pgp}

\subsection{S/MIME}
\index{\gls{smime}}
\label{sec:thg:dsv:smime}

%%%------------------------------------

\section{Signaturen in XML}
\index{Signaturen in XML}
\label{sec:thg:xml}


\subsection{XML-DSIG}
\index{Signaturen in XML!\gls{xml-dsig}}
\label{sec:thg:xml:dsig}
