% vim: set tw=160:

\chapter{Theoretische Grundlagen}
\index{Theoretische Grundlagen}
\label{chp:TheoretischeGrundlagen}

\section{Definitionen}

\subsection{Digitale Signaturen}
Unter dem kryptographischen Sammelbegriff der digitalen Signatur versteht man eine zuverlässige und nicht fälschbare Kontrollinformation die über den Urheber
und mögliche Modifikationen des Inhalts der zu schützenden Daten Auskunft gibt\cite[S. 28 ff]{kits}. Dabei kann eine digitale Signatur analog zu der Unterschrift
einer Person betrachtet werden\cite[S. 4]{esig:bsi}, jedoch mit erweiterten Eigenschaften.

\subsection{Bereiche in XHTML-Dokumenten}
Der Dokumenttyp \gls{xhtml} is ein Subset von \gls{xml}\cite{xhtml:w3c}\cite[S. 500 ff]{xhtml:oreilly} und kann als solches in Form eines \gls{dom}-Baumes betrachtet werden. Ein Bereich innerhalb eines
solchen Dokuments kann also sowohl aus einer hierachisch gegliederten Menge der Elemente eines der Zweige dieses Baums verstanden werden. Als weitere Definition
bietet sich eine nicht der ursprünglichen Hierarchie folgende Menge an Elementen aus dem Baum an. Hierbei werden einzelne Elemente anhand definierter
Filterkriterien in einer ungeordneten Menge abgelegt.

%%%------------------------------------

\section{Digitale Signaturverfahren}
\index{Digitale Signaturverfahren}
\label{sec:thg:dsv}


\subsection{PGP}
\index{\gls{pgp}}
\label{sec:thg:dsv:pgp}

\subsection{S/MIME}
\index{\gls{smime}}
\label{sec:thg:dsv:smime}

%%%------------------------------------

\section{Signaturen in XML}
\index{Signaturen in XML}
\label{sec:thg:xml}


\subsection{XML-DSIG}
\index{Signaturen in XML!\gls{xml-dsig}}
\label{sec:thg:xml:dsig}
