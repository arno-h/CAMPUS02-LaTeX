% vim: set tw=160:

\chapter{Grundlagen und Definitionen}
\index{Grundlagen und Definitionen}
\label{chp:GrundlagenDefinitionen}

Um bei einer Arbeit, die sich mit mehreren Bereichen der Informatik beschäftigt, klare Aussagen treffen zu können, ist es unerlässlich, Kernbegriffe klar zu
definieren und sie somit von möglichen interdisziplinären Missverständnissen zu befreien.

\section{Webseiten}
\index{Webseiten}
\label{sec:GrundlagenDefinitionen:Webseiten}
Eine Webseite repräsentiert eine Menge von Dateien, die unter dem gemeinsamen Namensraum einer Domain abrufbar sind.\\
Zu diesen Dateien können \gls{html}- und \gls{xhtml}-Dokumente gehören, sowie auch Bild-, Ton-, und Videodateien.

\section{Bereiche in XHTML-Dokumenten}
\index{Bereiche in XHTML-Dokumenten}
\label{sec:GrundlagenDefinitionen:BereicheXHTML}
Der Dokumententyp \gls{xhtml} ist eine Anwendungsform von \gls{xml}\cite{xhtml:w3c} und kann dadurch in ihrer Form als \gls{dom}-Baum betrachtet
werden\cite{xhtml:oreilly}. Da moderne \Gls{webbrowser} unterschiedliche Technlogien wie \gls{css} und \gls{javascript} zur ergänzenden Darstellung von
\gls{xhtml} nutzen, ist nicht sichergestellt, dass ein Zweig in einem \gls{xhtml}-Dokument einen geschlossenen Bereich in der grafischen Darstellung des
selben Dokumentes einnimmt. Umgekehrt können unterschiedliche Zweige aus dem \gls{dom}-Baum bei der Darstellung zu einem geschlossenen Bereich zusammengeführt
werden.\\

Als ein Bereich wird im Rahmen dieser Arbeit die Positionierung eines oder mehrerer Elemente innerhalb eines \gls{dom}-Baumes definiert, nicht die grafische
Darstellung in einem \gls{webbrowser}.

\section{Digitale Signaturen}
\index{Digitale Signaturen}
\label{sec:GrundlagenDefinitionen:DigitaleSignaturen}
Unter dem kryptographischen Sammelbegriff der digitalen Signatur versteht man eine zuverlässige und nicht fälschbare Kontrollinformation die über den Urheber
und mögliche Modifikationen des zu schützenden Inhalts Auskunft gibt\cite{kits}. Dabei kann eine digitale Signatur analog zu der Unterschrift einer Person
betrachtet werden\cite{esig:bsi}, jedoch mit erweiterten Eigenschaften.\\
Der Begriff der digitalen Signatur wird als eine Information definiert, welche dieIdentität einer Urheber-Entität mit einer digitalen Information (Nachricht)
assoziiert und die drei kryptografischen Sicherheitsdienste Integrität, Authentizität und Verbindlichkeit erfüllt\cite{hac}.

\subsection{Integrität}
\index{Digitale Signaturen!Integrität}
\label{sec:GrundlagenDefinitionen:DigitaleSignaturen:Integrität}
Durch Integrität wird garantiert, dass eine Nachricht nicht verändert wurde\cite{niag}. Der Empfänger der Nachricht ist damit in der Lage eine Aussage über
mögliche Änderungnen am Inhalt zu treffen.

\subsection{Authentizität}
\index{Digitale Signaturen!Authentizität}
\label{sec:GrundlagenDefinitionen:DigitaleSignaturen:Authentizität}
Der Empfänger einer Nachricht kann durch die Authentizität deren Urheber feststellen. Der Urheber der Nachricht weisst dabei seine Identität seinem
Partner gegenüber durch ein Geheimnis oder eine Fertigkeit aus, die ihn für den Empfänger eindeutig kennzeichnet. Die Authentizität einer Nachricht setzt ihre
Integrität voraus, denn eine veränderte Nachricht ist auch nicht mehr authentisch\cite{kits}.

\subsection{Verbindlichkeit}
\index{Digitale Signaturen!Verbindlichkeit}
\label{sec:GrundlagenDefinitionen:DigitaleSignaturen:Verbindlichkeit}
Als Verbindlichkeit einer Nachricht wird die Nachweisbarkeit des Urhebers einer Nachricht gegenüber Dritten definiert. Dieser Sicherheitsdienst schließt die
Eigenschaften der Authentizität und der Integrität mit ein\cite{kits}.

\section{Digitale Signaturverfahren}
\index{Digitale Signaturverfahren}
\label{sec:GrundlagenDefinitionen:DigitaleSignaturen:Verfahren}
Ein digitales Signaturverfahren stellt ein Verfahren dar, mit welchem eine Entität ihre Identität an eine Information binden kann. Der Vorgang des Signierens
wird durch das Transformieren der Information und eines Geheimnisses, das nur dem Urheber gekannt ist, in eine Signatur definiert\cite{hac}.

\subsection{OpenPGP}
\index{OpenPGP}
\label{sec:GrundlagenDefinitionen:DigitaleSignaturen:Verfahren:openpgp}
\gls{openpgp} wurde im Jahr 1991 von Philip Zimmermann\cite{zimmermann:pgp} entwickelt, mit dem Ziel die digitale Kommunikation vor dem Zugriff durch Dritte zu
schützen. Eine der Motivationen während der Entwicklung waren die wiederholten Abhöraktionen der Bürgerrechtsbewegungen in den Vereinigten Staaten von Amerika
durch die dortigen Geheimdienste\cite{singh:messages}.

Der Name der aktuelle Spezifikation für das Datenformat ist \gls{openpgp} und ist im RFC4880\cite{openpgp:ietf} festgelegt. Dabei werden zwei Betreibsarten festgelegt:
\begin{itemize}
    \item Signieren
    \item Verschlüsseln
\end{itemize}

\subsubsection{Signaturen}
Im Rahmen dieser Arbeit wird nur auf den Signatur-Betreib eingegangen. \gls{openpgp} bildet dabei für eine zu signierende Nachricht den passenden Hashwert und
verschlüsselt diesen mit dem privaten Schlüssel. Diese verschlüsselte Information stellt nun die Signatur zur Nachricht dar und wird dieser beigefügt.
Der Empfänger entschlüsselt die Signaturinformation, bildet den Hashwert über die signierte Nachricht und vergleicht diesen mit dem entschlüsselten Hashwert.
Sind beide miteinander ident, ist die Signatur gültig und die Integrität der Nachricht gewährleistet.
\cite{hac}\cite{singh:messages}

\subsubsection{Web Of Trust}
Um die Ermittlung der Authentizität einer Signatur zu erleichtern stellt \gls{openpgp} eine Funktionalität zur Beglaubigung von Schlüsseln zur Verfügung. Dabei
werden die Schlüssel von anderen \gls{openpgp}-Benutzern signiert. Das ermöglicht es, Rückschlüsse auf die Vertrauenswürdigkeit von Schlüsseln zu ziehen, ohne
von einer hierachischen Struktur abhängig zu sein.\\

\begin{figure}
\centering
\begin{tikzpicture}[node distance   = 4 cm]
    \tikzset{Key/.style = {shape          = rectangle,
                           rounded corners,
                           drop shadow,
                           fill           = orange,
                           draw           = black,
                           thick,
                           text           = black,
                           text centered,
                           inner sep      = 3pt,
                           outer sep      = 0pt,
                           minimum size   = 24 pt}}
    \tikzset{DirectTrust/.style   = {thick,
                                     double          = green,
                                     double distance = 1pt}}
    \tikzset{IndirectTrust/.style   = {thin,
                                       double          = blue,
                                       double distance = 1pt}}
    \tikzset{Label/.style =   {draw,
                               fill           = white,
                               text           = black}}
    \node[Key](A){Alice};
    \node[Key,right=of A](B){Bob};
    \node[Key,right=of B](C){Carmen};
    \node[Key,above=of B](D){Dorian};     
    \draw[DirectTrust](B) to node[Label]{vertraut} (D) ;
    \tikzset{DirectTrust/.append style = {bend left}}
    \draw[DirectTrust](A) to node[Label]{vertraut} (B);
    \draw[DirectTrust](B) to node[Label]{vertraut} (C);
    \draw[DirectTrust](D) to node[Label]{vertraut} (C);
    \tikzset{IndirectTrust/.append style = {bend left}}
    \draw[IndirectTrust](A) to node[Label]{vertraut indirekt} (D) ;
    \tikzset{IndirectTrust/.append style = {bend right}}
    \draw[IndirectTrust](A) to node[Label]{vertraut indirekt} (C) ;

\end{tikzpicture}

\caption{Schematische Darstellung des Web-Of-Trust}
\label{fig:wot}
\end{figure}

Wenn Alice den Schlüssel von Bob signiert, dann ermöglicht dies einer Person, die bereits dem Schlüssel von Alice vertraut, auch dem Schlüssel von Bob zu
vertrauen. Durch diese Delegation von Vertrauen bildet sich eine flache, vernetzte Struktur die unter dem Namen \gls{wot} bekannt ist. Jeder Benutzer von
\gls{openpgp} ist somit in der Lage, sein Vertrauen in Schlüssel sehr granular und spezifisch zu vergeben.

\subsection{X.509}
\index{\gls{x509}}
\label{sec:GrundlagenDefinitionen:DigitaleSignaturen:Verfahren:x509}

\section{XML-DSig}
\index{\gls{xml-dsig}}
\label{sec:GrundlagenDefinitionen:xml-dsig}
