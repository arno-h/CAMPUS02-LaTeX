% vim: set tw=160:

\chapter{Grundlagen und Definitionen}
\index{Grundlagen und Definitionen}
\label{chp:GrundlagenDefinitionen}

Um bei einer Arbeit, die sich mit mehreren Bereichen der Informatik beschäftigt, klare Aussagen treffen zu können, ist es unerlässlich, Kernbegriffe klar zu
definieren und sie somit von möglichen interdisziplinären Missverständnissen zu befreien.

\section{Webseiten}
\index{Webseiten}
\label{sec:GrundlagenDefinitionen:Webseiten}
Eine Webseite representiert eine Menge von Dateien, die unter dem gemeinsamen Namensraum einer Domain abrufbar sind. Zu diesen Dateien gehören \gls{html}- und
\gls{xhtml}-Dokumente, sowie auch Bild-, Ton-, und Videodateien.

\section{Bereiche in XHTML-Dokumenten}
\index{Bereiche in XHTML-Dokumenten}
\label{sec:GrundlagenDefinitionen:BereicheXHTML}
Der Dokumenttyp \gls{xhtml} is eine Anwendungsform von \gls{xml}\cite{xhtml:w3c} und kann als solcher in Form eines \gls{dom}-Baumes
betrachtet werden\cite{xhtml:oreilly}. Da moderne \Gls{webbrowser} unterschiedliche Technlogien wie \gls{css} und \gls{javascript} zur ergänzenden Darstellung von
\gls{xhtml} nutzen, ist nicht sichergestellt, dass ein Zweig in einem \gls{xhtml}-Dokument einen geschlossenen Bereich in der grafischen Darstellung des
Dokumentes einnimmt. Umgekehrt können unterschiedliche Zweige aus dem \gls{dom}-Baum bei der Darstellung zu einem geschlossenen Bereich zusammengeführt werden.\\
Als ein Bereich wird im Rahmen dieser Arbeit die Positionierung eines oder mehrerer Elemente innerhalb eines \gls{dom}-Baumes definiert, nicht die grafische
Darstellung in einem \gls{webbrowser}.

\section{Digitale Signaturen}
\index{Digitale Signaturen}
\label{sec:GrundlagenDefinitionen:DigitaleSignaturen}
Unter dem kryptographischen Sammelbegriff der digitalen Signatur versteht man eine zuverlässige und nicht fälschbare Kontrollinformation die über den Urheber
und mögliche Modifikationen des Inhalts der zu schützenden Daten Auskunft gibt\cite{kits}. Dabei kann eine digitale Signatur analog zu der Unterschrift
einer Person betrachtet werden\cite{esig:bsi}, jedoch mit erweiterten Eigenschaften.\\
Der Begriff der digitalen Signatur wird als .... definiert welche die folgenden drei kryptografischen Sicherheitsdienste erfüllen kann.

\subsection{Authentizität}

\subsection{Integrität}

\subsection{Verbindlichkeit}

\section{Digitale Signaturverfahren}
\index{Digitale Signaturverfahren}
\label{sec:GrundlagenDefinitionen:DigitaleSignaturen:Verfahren}

\subsubsection{PGP}
\index{\gls{pgp}}
\label{sec:GrundlagenDefinitionen:DigitaleSignaturen:Verfahren:pgp}

\subsubsection{S/MIME}
\index{\gls{smime}}
\label{sec:GrundlagenDefinitionen:DigitaleSignaturen:Verfahren:smime}

\section{Signaturen mit XML-DSig}
\index{Signaturen mit \gls{xml-dsig}}
\label{sec:GrundlagenDefinitionen:xml-dsig}
