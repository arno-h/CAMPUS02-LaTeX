% vim: set tw=160:

\chapter{Grundlagen und Definitionen}
\index{Grundlagen und Definitionen}
\label{chp:GrundlagenDefinitionen}

Um bei einer Arbeit, die sich mit mehreren Bereichen der Informatik beschäftigt, klare Aussagen treffen zu können, ist es unerlässlich, Kernbegriffe klar zu
definieren und sie somit von möglichen interdisziplinären Missverständnissen zu befreien.

\section{Webseiten}
\index{Webseiten}
\label{sec:GrundlagenDefinitionen:Webseiten}
Eine Webseite repräsentiert eine Menge von Dateien, die unter dem gemeinsamen Namensraum einer Domain abrufbar sind. Zu diesen Dateien können \gls{html}- und
\gls{xhtml}-Dokumente gehören, sowie auch Bild-, Ton-, und Videodateien.

\section{Bereiche in XHTML-Dokumenten}
\index{Bereiche in XHTML-Dokumenten}
\label{sec:GrundlagenDefinitionen:BereicheXHTML}
Der Dokumenttyp \gls{xhtml} is eine Anwendungsform von \gls{xml}\cite{xhtml:w3c} und kann dadurch in ihrer Form als \gls{dom}-Baum betrachtet
werden\cite{xhtml:oreilly}. Da moderne \Gls{webbrowser} unterschiedliche Technlogien wie \gls{css} und \gls{javascript} zur ergänzenden Darstellung von
\gls{xhtml} nutzen, ist nicht sichergestellt, dass ein Zweig in einem \gls{xhtml}-Dokument einen geschlossenen Bereich in der grafischen Darstellung des
selben Dokumentes einnimmt. Umgekehrt können unterschiedliche Zweige aus dem \gls{dom}-Baum bei der Darstellung zu einem geschlossenen Bereich zusammengeführt
werden.\\

Als ein Bereich wird im Rahmen dieser Arbeit die Positionierung eines oder mehrerer Elemente innerhalb eines \gls{dom}-Baumes definiert, nicht die grafische
Darstellung in einem \gls{webbrowser}.

\section{Digitale Signaturen}
\index{Digitale Signaturen}
\label{sec:GrundlagenDefinitionen:DigitaleSignaturen}
Unter dem kryptographischen Sammelbegriff der digitalen Signatur versteht man eine zuverlässige und nicht fälschbare Kontrollinformation die über den Urheber
und mögliche Modifikationen des zu schützenden Inhalts Auskunft gibt\cite{kits}. Dabei kann eine digitale Signatur analog zu der Unterschrift einer Person
betrachtet werden\cite{esig:bsi}, jedoch mit erweiterten Eigenschaften.\\
Der Begriff der digitalen Signatur wird als .... definiert, welche die folgenden drei kryptografischen Sicherheitsdienste erfüllen kann.

\subsection{Integrität}
\index{Digitale Signaturen!Integrität}
\label{sec:GrundlagenDefinitionen:DigitaleSignaturen:Integrität}
Durch Integrität wird garantiert, dass eine Nachricht nicht verändert wurde\cite{niag}.

\subsection{Authentizität}
\index{Digitale Signaturen!Authentizität}
\label{sec:GrundlagenDefinitionen:DigitaleSignaturen:Authentizität}
Der Empfänger einer Nachricht kann durch die Authentizität deren Urheber feststellen. Der Urheber der Nachricht weisst dabei seine Identität seinem
Partner gegenüber durch ein Geheimnis oder eine Fertigkeit aus, die ihn für den Empfänger eindeutig kennzeichnet. Die Authentizität einer Nachricht setzt ihre
Integrität voraus, denn eine veränderte Nachricht ist auch nicht mehr authentisch\cite{kits}.

\subsection{Verbindlichkeit}
\index{Digitale Signaturen!Verbindlichkeit}
\label{sec:GrundlagenDefinitionen:DigitaleSignaturen:Verbindlichkeit}
Verbindlichkeit einer Nachricht bedeuted, dass auch gegenüber Dritten eindeutig nachgewiesen werden kann, wer der Autor einer Nachricht war. Die Eigenschaft der
Verbindlichkeit schließt die Eigenschaften der Authentizität und der Integrität mit ein\cite{kits}.

\section{Digitale Signaturverfahren}
\index{Digitale Signaturverfahren}
\label{sec:GrundlagenDefinitionen:DigitaleSignaturen:Verfahren}

\subsection{PGP}
\index{\gls{pgp}}
\label{sec:GrundlagenDefinitionen:DigitaleSignaturen:Verfahren:pgp}

\subsection{X.509}
\index{\gls{x509}}
\label{sec:GrundlagenDefinitionen:DigitaleSignaturen:Verfahren:x509}

\section{XML-DSig}
\index{\gls{xml-dsig}}
\label{sec:GrundlagenDefinitionen:xml-dsig}
