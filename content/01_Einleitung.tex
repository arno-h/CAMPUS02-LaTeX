%!TEX root =  ../diplomarbeit.tex
%%%%%%%%%%%%%%%%%%%%%%%%%%%%%%%%%%%%%%%
%%%     KAPITEL 1
%%%     EINLEITUNG
%%%     2007/10/17
%%%%%%%%%%%%%%%%%%%%%%%%%%%%%%%%%%%%%%%
\chapter{Einleitung}
\index{Einleitung}%
\label{chp:einleitung}%


%%%------------------------------------
%%%---- AUSGANGSSITUATION -------------
%%%------------------------------------

\section{Ausgangssituation}
\index{Ausgangssituation}%
\label{sec:einl:ausgangssituation}%

In kleinen Softwarebetrieben kommt es immer wieder vor, dass neue Mitarbeiter (auch ohne Berufserfahrung), auf Grund des Ressourcenmangels, kurz nach ihrem Eintritt, eigene Projekte mehr oder minder alleine bearbeiten m�ssen. Die ersten Projekte k�nnen die Mitarbeiter dabei jedoch zerm�rben da noch keine Einsicht in diese Materie vorhanden ist. Aus diesem Grund sollte in jeder Firma ein einfach anzuwendendes Vorgehen dokumentiert sein, welches angewendet werden kann um gr��ere Schwierigkeiten zu vermeiden. Dies kann jedoch nur funktionieren wenn der Prozess durchgehend dokumentiert ist und dabei unterst�tzend aus einem Pool von bereits abgeschlossenen und aktiven Projekten Synergien gezogen werden k�nnen.  \\
Das Problem dabei ist, dass dies in kleinen Unternehmen wegen der ?�berschaubarkeit? bei bis zu zw�lf Mitarbeitern oft vernachl�ssigt wird und in gro�en Unternehmen wegen der zu hohen Komplexit�t keinen Anklang findet. Zus�tzlich kommt der Zeitdruck, welcher schnelle Resultate erzwingt und die Dokumentation oft Au�en vorl�sst (diese wird dann kaum mehr bzw. nicht mehr in der gew�nschten Qualit�t nachgeholt). Durch schlechte Ziel und Anforderungsformulierungen wird wegen zeitintensiven Fehlentscheidungen dieser Zeitdruck noch verst�rkt.

%%%------------------------------------
%%%---- AUFGABENSTELLUNG --------------
%%%------------------------------------
\section{Aufgabenstellung}
\index{Aufgabenstellung}%
\label{sec:einl:aufgabenstellung}%

Die prim�re Aufgabenstellung dieser Arbeit besteht darin zu bestimmen, welche Aufgaben und Informationen Projektportfoliomanagement bearbeitet um einem Unternehmen von Nutzen zu sein bzw. keine unn�tigen Verwaltungst�tigkeiten zu erzeugen. \\
Dabei soll vor Allem auf folgende Punkte eingegangen werden:
\begin{itemize}
  \item Auf welchem Abstraktionslevel wird gearbeitet?
  \item Welche Informationen werden verarbeitet?
  \item Welcher Nutzen wird erzielt?
  \item Ist es f�r die Aufwandsabsch�tzungen dienlich?
\end{itemize}
Des Weiteren soll die Definition eines Gesch�ftsprozesses exemplarisch an einer Anforderungsanalyse f�r ein Projektportfoliomanagementsystem durchgef�hrt werden.

%%%------------------------------------
%%%---- ZIELSETZUNG -------------------
%%%------------------------------------
\section{Zielsetzung der Arbeit}
\index{Ziele}%
\label{sec:einl:ziele}%

Diese Arbeit wird vor allem die Thematik des Projektportfoliomanagement behandeln sowie dessen Auswirkungen auf das Unternehmen.\\
Nicht behandelt wird die Erstellung eines neuen Anforderungsanalysemodells, stattdessen wird sich der Prozess an bestehende Systeme anlehnen. \\
Des Weiteren wird sich diese Arbeit nicht mit der Einf�hrung von Gesch�ftsprozessdokumentationen besch�ftigen sondern nur mit der Definition eines weiteren Prozesses, in diesem Fall der Anforderungsanalyse.

%%%------------------------------------
%%%---- VORGEHEN ----------------------
%%%------------------------------------
\section{Vorgehen und Methodik}
\index{Methodik}%
\label{sec:einl:methodik}%


%%%------------------------------------
%%%---- AUFBAU  -----------------------
%%%------------------------------------
\section{Aufbau der Arbeit}
\index{Aufbau der Arbeit}%
\label{sec:einl:aufbau}%
