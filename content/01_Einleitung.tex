% vim: set tw=160:

\chapter{Einleitung}
\index{Einleitung}%
\label{chp:Einleitung}

\section{Ausgangssituation}
\index{Ausgangssituation}%
\label{sec:Einleitung:ausgangssituation}
Kryptographische Merkmale wie Vertraulichkeit, Authentizität und Integrität von digitalen Inhalten gewinnen in einem wachsenden globalen Netzwerk immer
stärker an Bedeutung. Dies lässt sich anhand der Verbreitung von \gls{ssl}-fähigen Webservern und den darauf eingesetzten Zertifikaten bestimmen\cite{ssliverse:eff}.\\

\gls{ssl} dient jedoch nur der Vertraulichkeit sowie der Integrität der Gesmatheit aller übertragenen Daten und der Authentizität zwischen den Peers einer unidirektionalen
Komunikation, wie sie z.B. beim \gls{https} zur Anwendung kommt\cite{kits}. Aktive (umleiten der Verbindung) und passive Angriffe (Lesen der übertragenen Daten)
lassen sich dadurch verhindern, eine Prüfung der Integrität und Authentizität einzelner, festlegbarer Bereiche des übertragenen Inhalts ist damit jedoch nicht möglich.\\

In anderen Bereichen der Kommunikation wie Email und \gls{im} gehören Verfahren zum Nachweis der Identität des Absenders und der Integrität der Nachricht
bereits zum Leistungsumfang vieler Anwendungsprogramme\cite{2719799020071101}. Die Verfahren, welche dabei am häufigsten zum Einsatz kommen, sind \gls{pgp} und
\gls{smime}. Sie bieten die Möglichkeit, digitale Signaturen auf Basis von asymmetrischen Verschlüsselungsverfahren zu erstellen\cite{kits}. Dabei wird vom
Verfasser der Nachricht diese mit seiner digitalen Signatur versehen. Der Empfänger ist damit in der Lage, die in der Nachricht signierten Teile zu verifizieren.\\

Mit dem Aufkommen des sog. Web 2.0 entstand auf vielen Webseiten für die Besucher die Möglichkeit, eigene Inhalte einzubinden oder bestehende Inhalte weiter zu
verwenden. Der Betreiber einer \gls{website} wurde dadurch vom alleinigen Bereitsteller zu einem Verwalter von Informationen. Für Besucher solcher Seiten mit
gemischten redaktionellen und Benutzer-generierten Inhalten existert noch keine Möglichkeit, diese Trennung auch kryptografisch sicher nachvollziehen zu können.
Das Vertrauen eines Besuchers in die Identität des Urhebers und die Integrität der Inhalte kann deshalb nur über die optische Darstellung gebildet werden. Eine
optische Trennung von Inhalten bietet jedoch noch keine gesicherte Information über Authentizität und Integrität. Hier bieten sich digitale Signaturen  an, um
genau dieses Bedürfnis nach kryptografisch sicherer Auszeichnung von Inhalten einer \gls{website} erfüllen.

\section{Aufgabenstellung}
\index{Aufgabenstellung}%
\label{sec:Einleitung:aufgabenstellung}
Die primäre Aufgabenstellung dieser Arbeit besteht darin, zu bestimmen, welche Anforderungen an eine Integration von digitalen Signaturen in die
\gls{xhtml}-Dokumente einer \gls{website} gestellt werden. Folgende Aspekte sollen deshalb in dieser Arbeit erläutert werden:
\begin{itemize}
    \item Wie können digitale Signaturen in \gls{xhtml} integriert werden, ohne dass der entsprechende W3C-Standard\cite{xhtml:w3c} erweitert oder angepasst
    werden?
    \item Welche Erweiterungen von \gls{xml} eignen sich zur wahlfreien Selektion von Bereichen innerhalb eines \gls{xhtml}-Dokuments?
    \item Wie könenn die öffentlichen Schlüssel zu den Signaturen verteilt werden? Welche der bestehenden Technologien eignen sich dafür am besten?
\end{itemize}

\section{Zielsetzung der Arbeit}
\index{Ziele}
\label{sec:Einleitung:ziele}
Diese Arbeit wird die Eignung der vorhandenen Signaturverfahren für die \gls{xml}, sowie die damit einhergehenden Möglichkeiten zur Schlüsselverteilung
behandeln. Um den praktischen Nutzen eines solchen Verfahrens zu demonstrieren soll auch aufgezeigt werden, welche Arten von Angriffen auf Inhalte durch
digitale Signaturen in \gls{xhtml} verhindert werden können.\\

Nicht behandelt wird die Erstellung eines neuen Signaturverfahrens, eines neuen Signaturmodells für \gls{xml} oder die Erweiterung des bestehenden
\gls{xhtml}-Standards. Auch ist eine Implementierung eines Browser-Plugins nicht im Rahmen dieser Arbeit vorgesehen.

\section{Vorgehen und Methodik}
\index{Methodik}
\label{sec:Einleitung:methodik}


\section{Aufbau der Arbeit}
\index{Aufbau der Arbeit}
\label{sec:Einleitung:aufbau}
Eine kurze Einführung in \gls{xml-dsig} und verwandte Verfahren zur Signierung von Inhalten, danach Überleitung zur konkreten Arbeitsweise von \gls{xml-dsig}.
Vorstellung der Möglichkeiten zur Verbindung zwischen \gls{xml-dsig} und \gls{xhtml}. Evaluierung und Auswahl des am besten geeigneten Mechanismus, bewertet an
Praktikabilität und Sicherheit. Überblick über die von \gls{xml-dsig} unterstützten Signaturverfahren. Evaluierung der Verfahren und Auswahl eines Verfahrens
anhand der Eignung für den Einsatz in einem \gls{webbrowser}.

