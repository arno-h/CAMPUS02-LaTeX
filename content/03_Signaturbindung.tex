% vim: set tw=160:

\chapter{Signaturen mit XML-DSig}
\index{Signaturen mit XML-DSig}
\label{chap:SignaturenXMLDSig}

\section{Struktur einer Signatur}

\section{Referenzen}
\index{Referenzen}
\label{sec:Signaturbindung:Referenzen}

\section{Transformationen}
\index{Transformationen}
\label{sec:Signaturbindung:Transformationen}

\subsection{XPath}
\index{XPath}
\label{sec:Signaturbindung:Transformationen:XPath}

\subsection{XSLT}
\index{XSLT}
\label{sec:Signaturbindung:Transformationen:XSLT}

\section{Schlüsselinformation}

\section{Bindung an ein XHTML-Dokument}
Eine Signatur mit \gls{xml-dsig} hält ihre Verweise zu den signierten Inhalten über die Reference-Elemente. Im Fall der Verwendung innerhalb eines
\gls{webbrowser} muss das aufgerufene \gls{xhtml}-Dokument einen Verweis auf die Signatur beinhalten. Dies ist notwendig, da die Information über die Bereiche,
welche signiert wurden, nur in der Signatur selbst festgehalten werden. Ruft ein \gls{webbrowser} ein \gls{xhtml}-Dokument ab, so muss ihm bekannt gemacht
werden, dass für dieses Dokument Signaturinformationen vorhanden sind. Sind diese Informationen verfügbar gemacht worden, kann der \gls{webbrowser} daraus die
signierten Bereiche ermitteln.\\

Um solche Verweise auf Signaturen an die \gls{xhtml} binden zu können, bieten sich mehrere Verfahren an.

\subsection{HTTP-Header}
Eigener X-HTTP-Header wird vom Webserver ausgeliefert, worin über \glspl{url} auf Signatur-Dateien verwiesen wird. Keine Auswirkung auf XHTML, jedoch praktische Beschränkung
auf Anzahl der \glspl{url}?

\subsection{\xmlelem{link}-Element}
Der \gls{xhtml}-Standard definiert das \xmlelem{link}-Element, um Relationen zu anderen Dokumenten zu definieren \cite{xhtml:oreilly}. Es ist Teil des
\xmlelem{head}-Bereichs
und kann dort mehrfach eingesetzt werden. Ein einzelnes \xmlelem{link}-Element definiert genau eine Relation. Bezüglich des Verhaltens der \gls{webbrowser} beim
Verarbeiten eines \xmlelem{link}-Elements macht der \gls{xhtml}-Standard keine verbindlichen Aussagen. Es wird aber darauf hingewiesen, dass dieses Element speziell für
zukünftige, noch nicht definierte Verhaltensmuster bei \gls{webbrowser} vorgesehen ist.\\

Es ist damit möglich, über das \xmlattr{href}-Attribut des \xmlelem{link}-Elements den Verweis auf die \gls{url} einer Signaturinformation zu realisieren. Da das
Element auch mehrfach in einem \gls{xhtml}-Dokument verwendet werden kann, lassen sich damit auch mehrere unterschiedliche Signaturinformationen einbinden.

\lstinputlisting[language=HTML,caption={Verwendung von \xmlelem{link}},label=lst:link,emph={link,href},emphstyle=\color{red}\bfseries\emph]{source/link.html}

\subsection{XML-Namespaces}
Direktes Einbetten der Signatur in das Dokument selbst.

\lstinputlisting[language=HTML,caption={Verwendung von XML-Namespaces},label=lst:xmlns,emph={Signature},emphstyle=\color{red}\bfseries\emph]{source/xmlns.html}

\subsection{XInclude}
XInclude ist eine neue Technologie, die vom W3C entwickelt wurde, um mehrere \gls{xml}-Dokumente in ein gemeinsames Dokument zusammenzuführen \cite{xml:oreilly}.
Dabei wird das \xmlelem{include}-Element aus dem reservieren XML-Namespace für \gls{xinclude} benutzt um über \gls{uri} externe \gls{xml}-Dokumente während der
Auswertung des Dokuments einzubinden.

\lstinputlisting[language=HTML,caption={Verwendung von XInclude},label=lst:xinclude,emph={include,xi},emphstyle=\color{red}\bfseries\emph]{source/xinclude.html}
