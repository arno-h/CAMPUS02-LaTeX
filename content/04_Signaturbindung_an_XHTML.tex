% vim: set tw=160:

\chapter{Bindung an ein XHTML-Dokument}
\label{chap:Signaturbindung}
Eine Signatur mit \gls{xml-dsig} hält ihre Verweise zu den signierten Inhalten über die \xmlelem{Reference}-Elemente aus \fref{sec:XML-DSig:Referenzen}.
Im Fall der Verwendung innerhalb eines \gls{webbrowser} muss das aufgerufene \gls{xhtml}-Dokument einen Verweis auf die Signatur beinhalten. Dies ist notwendig,
da die Information über die Bereiche, welche signiert wurden, nur in der Signatur selbst festgehalten werden. Ruft ein \gls{webbrowser} ein \gls{xhtml}-Dokument
ab, so muss ihm bekannt gemacht werden, dass für dieses Dokument Signaturinformationen vorhanden sind. Sind diese Informationen verfügbar gemacht worden, kann
der \gls{webbrowser} daraus die signierten Bereiche ermitteln.

Um solche Verweise auf Signaturen an ein \gls{xhtml}-Dokument binden zu können, bieten sich mehrere Verfahren an.

\section{HTTP-Header}
\index{HTTP-Header}
Als Antwort auf die Anfrage eines \gls{webbrowser} wird vom Webserver ein eigener \gls{http}-Header ausgeliefert, welcher eine Liste von \glspl{url} enthält, die auf
Signatur-Dateien verweisen. Dadurch wäre eine klare Trennung der Signaturen vom \gls{xhtml}-Inhalt gewährleistet.
Ein \gls{http}-Header kann auch mehrfach in der Antwort eines Webservers an einen \gls{webbrowser} enthalten sein. \gls{http} sieht keine Größenbeschränkung für
die Header-Informationen vor \cite{http:ietf}, jedoch sind Beschränkungen in \gls{http}-verarbeitenden Programmen möglich.

\lstinputlisting[caption={Verwendung von \protect\gls{http}-Header},
                 label=lst:http-header,
                 emph={X-Signature}]{source/http-header}

\section{\xmlelem{link}-Element}
Der \gls{xhtml}-Standard definiert das \xmlelem{link}-Element, um Relationen zu anderen Dokumenten zu definieren \cite{xhtml:oreilly}. Es ist Teil des
\xmlelem{head}-Bereichs und kann dort mehrfach eingesetzt werden. Ein einzelnes \xmlelem{link}-Element definiert genau eine Relation. Bezüglich des Verhaltens
der \gls{webbrowser} beim Verarbeiten eines \xmlelem{link}-Elements macht der \gls{xhtml}-Standard keine verbindlichen Aussagen. Es wird aber darauf
hingewiesen, dass dieses Element speziell für zukünftige, noch nicht definierte Verhaltensmuster bei \gls{webbrowser} vorgesehen ist.

Es ist damit möglich, über das \xmlattr{link}{href}-Attribut des \xmlelem{link}-Elements den Verweis auf die \gls{url} einer Signaturinformation zu realisieren. In
\fref{lst:link} ist dies auf \fref{lin:link-href} zu sehen. Da das
Element auch mehrfach in einem \gls{xhtml}-Dokument verwendet werden kann, lassen sich damit auch mehrere unterschiedliche Signaturinformationen einbinden.

\lstinputlisting[language=HTML,
                 caption={Verwendung von \xmlelem{link}},
                 label=lst:link,
                 emph={link,href}]{source/link.html}

\section{XML-Namespaces}
\label{sec:Signaturbindung:XML-NS}
Um mehrere verschiedene \gls{xml}-Dialekte in einem gemeinsamen Dokument zu trennen werden \gls{xml-ns} benutzt \cite{xml-ns:w3c}. Dies lässt sich auf die
beiden \gls{xml}-Dialekte \gls{xhtml} und \gls{xml-dsig} anwenden, um sie in einem Dokument zu kombinieren. Dazu kann die Signatur an einer beliebigen Stelle
innerhalb des \gls{xhtml}-Dokuments platziert werden und über \xmlelem{Reference} auf Bereiche des Dokuments verweisen. Zur Identifikation der einzelnen
Namespaces werden \glspl{uri} wie in \fref{tab:xml-namespaces} benutzt, welche in der Spezifikation der \gls{xml}-Dialekte festgelegt sind. 

\fref{lst:xmlns} zeigt die Kombination der Namespaces von \gls{xhtml} und \gls{xml-dsig}. Dabei wird für das Wurzel-Element des Dokuments, \xmlelem{html} auf
\fref{lin:xmlns-html}, der Namespace auf den von \gls{xhtml} gesetzt und somit auf alle darunter gelegenen Elemente vererbt. Erst beim Signatur-Element,
\xmlelem{Signature} auf \fref{lin:xmlns-signature}, wird der Namespace auf den von \gls{xml-dsig} geändert, wobei nun wiederrum dieser Namespace auf die unter
der Deklaration gelegenen Elemente vererbt wird. Es ist auch möglich, einen Namespace auf ein Element explizit ohne Vererbung zuzuweisen, da aber der Aufbau
einer \gls{xml-dsig}-Signatur immer aus mehreren Elementen aus dem selben Namespace besteht, ist dies hier nicht praktikabel und soll nicht näher erläutert
werden.

\begin{table}
    \centering
    \begin{tabularx}{\textwidth}{ l X }
        \gls{xml}-Dialekt  & \gls{xml-ns} \\
        \hline
        \hline
        \gls{xhtml} & \url{http://www.w3.org/1999/xhtml} \\
        \hline
        \gls{xml-dsig} & \url{http://www.w3.org/2000/09/xmldsig\#} \\
        \hline
        \gls{xinclude} & \url{http://www.w3.org/2001/XInclude} \\
        \hline
        \gls{xslt} & \url{http://www.w3.org/1999/XSL/Transform} \\
        \hline
    \end{tabularx}
    \caption{Namespaces für unterschiedliche \protect\gls{xml}-Dialekte}
    \label{tab:xml-namespaces}
\end{table}

\lstinputlisting[language=HTML,
                 caption={Verwendung von XML-Namespaces},
                 label=lst:xmlns,
                 emph={Signature}]{source/xmlns.html}

Ein Nachteil der \gls{xml-ns} ergibt sich aus der selben Situation die auch schon in \fref{sec:XML-DSig:Transformationen:XSLT} geschildert wurde. Die
Transformationen der Signatur müssen sicherstellen, dass die Elemente der Signatur selbst nicht zu den signierten Informationen gehört.

\section{XInclude}
\label{sec:Signaturbindung:XInclude}
\gls{xinclude} ist eine neue Technologie, die vom W3C entwickelt wurde, um mehrere \gls{xml}-Dokumente in ein gemeinsames Dokument zusammenzuführen
\cite{xml:oreilly}. Dabei wird das \xmlelem{include}-Element aus dem reservieren \gls{xml-ns} für \gls{xinclude} benutzt um über das
\xmlattr{xinclude}{href}-Attribut den \gls{uri} eines externen \gls{xml}-Dokuments während der Auswertung des Haupt-Dokuments einzubinden. Da das Dokument nach
dem Verarbeiten der \gls{xinclude}-Anweisungen wieder einen einzelnen \gls{xml}-Baum darstellt, ist es nötig wie in \fref{sec:Signaturbindung:XML-NS} die
Signatur mit ihrem eigenen \gls{xml-ns} von den \gls{xhtml}-Elementen abzugrenzen. Die Nachteile sind somit auch die selben wie bei der direkten Einbindung der
Signatur mittels \gls{xml-ns}.

\lstinputlisting[language=HTML,caption={Verwendung von XInclude},label=lst:xinclude,emph={include,xi}]{source/xinclude.html}


\section{Mikroformate}
Mikroformate sind auf Konvention beruhende, HTML-basierte Beschreibungen bestimmter Daten, zum Beispiel von Kontaktdaten. Mikroformate sind vor allem für
Webanwendungen wie etwa Suchmaschinen interessant, die aus den Mikroformaten datenfeldorientierte Informationen ziehen können \cite{mikroformate}.

Jede der zuvor vorgestellten Methoden macht es erforderlich zuerst das gesamte Dokument zu laden, um zu garantieren, dass alle Verweise aus den
\xmlelem{Reference}-Elementen auflösbar sind. Mit einem Mikroformat, das mit einer fest vorgegebenen \gls{xml-dsig}-Struktur arbeitet, könnten die verbleibenden
variablen Informationen aus der Struktur in eigene \gls{html}-Attribute eingebettet werden. Diese beiden Informationen, die nicht vorgegeben werden können sind
\xmlelem{KeyInfo} sowie \xmlelem{SignatureValue}. Um diese Attribute von \gls{xhtml} zu trennen müsste ein eigener \gls{xml-ns} geschaffen werden. Da ein
solches Mikroformat erst geschaffenwerden müsste, kann hier eine beliebige, global eindeutige \gls{uri} als Namespace benutzt werden.

Für den Rest der Informationen aus \gls{xml-dsig} wie \xmlelem{Reference} kann angenommen werden, dass es auf die Informationen innerhalb des
\gls{xhtml}-Elements verweist, in dem die Attribute des Mikroformats angehängt sind. Eine \xmlelem{Transformation} für \gls{c14n} und \gls{sha1} als
\xmlelem{DigestMethod} können gleichfalls vorgegeben werden. Dadurch kann, bereits während der \gls{webbrowser} das \gls{xhtml}-Dokument lädt, dynamisch eine
\gls{xml-dsig}-Struktur gebaut und auch ausgewertet werden. Dies hat den Vorteil, dass die Signaturen für Bereiche einer \gls{webseite} ausgewertet werden
können, bevor das gesamte \gls{xhtml}-Dokument vom \gls{webbrowser} geladen ist. Auch existiert so automatisch eine Zuordnung im \gls{webbrowser} zwischen
Signatur und signierten Elementen, ohne erst die \xmlelem{Reference}-Elemente auswerten zu müssen.

\lstinputlisting[language=HTML,
                 caption={Verwendung von Mikroformaten},
                 label=lst:mikroformat,
                 emph={signature,key,xml-dsig}]{source/mikroformate.html}

Ein Nachteil dieser Methode liegt darin, dass Teile der Flexibilität von \gls{xml-dsig} verloren gehen, da immer von vordefinierten \xmlelem{Reference}- und
\xmlelem{Transformation}-Elementen ausgegangen wird. Die Selektion von einzelnen Elementen je Signatur ist so nicht mehr möglich.

Eine zweite Variante von Mikroformaten ähnelt der in \fref{sec:Signaturbindung:XML-NS} beschriebenen Methode mit \gls{xml-ns}. Dabei wird ein externes
\gls{xml-dsig}-Dokument mit einem \gls{url} in einem Attribut aus einem eigenen \gls{xml-ns} referenziert. Dieses \gls{xml-dsig} bietet dann wieder die volle
Flexibilität, könnte aber so ausgelegt werden, dass nur Informationen selektiert und signiert werden können, die sich wie bei der ersten Variante unterhalb des
\gls{xhtml}-Elements mit dem entsprechenden Attribut befinden. Es wäre auch eine Kombination dieser beiden Varianten unter einem gemeinsamen Namespace
realisierbar.

