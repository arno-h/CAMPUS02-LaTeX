% vim: set tw=160:

% Acronyms
\newacronym{im}{IM}{Instant Messaging}
\newacronym{pgp}{PGP}{Pretty Good Privacy}
\newacronym{smime}{S/MIME}{Secure/Multipurpose Internet Mail Extensions}
\newacronym{ssl}{SSL}{Secure Sockets Layer}
\newacronym{pki}{PKI}{Public Key Infrastructure}
\newacronym{tls}{TLS}{Transport Layer Security}
\newacronym{xml}{XML}{eXtensible Markup Language}
\newacronym{xml-dsig}{XML-DSig}{XML Digital Signature}
\newacronym{html}{HTML}{Hypertext Markup Language}
\newacronym{xhtml}{XHTML}{eXtensible Hypertext Markup Language}
\newacronym{dom}{DOM}{Document Object Model}
\newacronym{http}{HTTP}{Hypertext Transfer Protocol}
\newacronym{https}{HTTPS}{Hypertext Transfer Protocol Secure}
\newacronym{css}{CSS}{Cascading Style Sheets}
\newacronym{uri}{URI}{Uniform Resource Identifier}
\newacronym{url}{URL}{Uniform Resource Locator}
\newacronym{wot}{WOT}{Web Of Trust}
\newacronym{itu}{ITU}{International Telecommunication Union}
\newacronym{ca}{CA}{Certificate Authority}
\newacronym{cot}{COT}{Chain Of Trust}
\newacronym{w3c}{W3C}{World Wide Web Consortium}
\newacronym{ietf}{IETF}{Internet Engineering Task Force}
\newacronym{asn1}{ASN.1}{Abstract Syntax Notation One}
\newacronym{ipsec}{IPsec}{Internet Protocol Security}
\newacronym{eap}{EAP}{Extensible Authentication Protocol}
\newacronym{dn}{DN}{Distinguished Name}
\newacronym{xml-ns}{XML-NS}{XML-Namespace}
\newacronym{rsa}{RSA}{Rivest, Shamir und Adleman}
\newacronym{dsa}{DSA}{Digital Signature Algorithm}
\newacronym{sha1}{SHA-1}{Secure Hash Algorithm One}
\newacronym{c14n}{C14N}{Canonicalization}
\newacronym{sgml}{SGML}{Standard Generalized Markup Language}
\newacronym{psm}{PSM}{Personal Security Manager}
\newacronym{nss}{NSS}{Network Security Service}
\newacronym{x509-ev}{EV}{Extended Validation}

%Glossary
\newglossaryentry{x509}{type=main, name={X.509v3}, description={Ein Standard zur Speicherung von kryptographischen Informationen für eine Public Key Infrastructure}}
\newglossaryentry{openpgp}{type=main, name={OpenPGP}, description={Ein Datenformat, um Informationen verschlüsselt zu speichern und digitale Signaturen zu erzeugen}}
\newglossaryentry{webbrowser}{type=main, name={Webbrowser}, description={Ein Programm zur Darstellung von Websites}}
\newglossaryentry{website}{type=main, name={Website}, plural={Websites}, description={Eine Menge von Dateien im WWW, die organisatorisch einer
Person/Organisation/Gruppe zugeordnet sind}}
\newglossaryentry{webseite}{type=main, name={Webseite}, plural={Webseiten}, description={Sammelbegriff für Dateien die gemeinsam dargestellt werden. Zum
Beispiel XHTML-Dokument mit Bildern, CSS und JavaScript-Elementen}}
\newglossaryentry{javascript}{type=main, name={JavaScript}, description={Eine Skriptsprache, die hauptsächlich in Web-Browsern eingesetzt wird}}
\newglossaryentry{xinclude}{type=main, name={XInclude}, description={Ein generischer Mechanismus zum Zusammenführen von XML-Dokumenten}}
\newglossaryentry{xpointer}{type=main, name={XPointer}, description={Eine Anfragesprache, um Teile eines XML-Dokumentes zu adressieren}, see={xpath}}
\newglossaryentry{xpath}{type=main, name={XPath}, description={Eine Abfragesprache, um Teile eines XML-Dokumentes zu adressieren}}
\newglossaryentry{xslt}{type=main, name={XSLT}, description={Eine Programmiersprache zur Transformation von XML-Dokumenten}}
\newglossaryentry{xsl}{type=main, name={XSL}, description={Eine Familie von Transformationssprachen zur Definition von Layouts für XML-Dokumente}}
\newglossaryentry{oktett}{type=main, name={Oktett}, plural={Oktetten}, description={Bezeichnung für eine geordnete Zusammenstellung von 8 Bit}}
\newglossaryentry{base64}{type=main, name={Base64}, description={Verfahren zur Kodierung von 8-Bit-Binärdaten, in eine Zeichenfolge, die nur aus lesbaren
Codepage-unabhängigen ASCII-Zeichen besteht}}
\newglossaryentry{firefox}{type=main, name={Firefox}, description={Ein freier Webbrowser des Mozilla-Projekts}}
\newglossaryentry{firefox-addon}{type=main, name={Firefox-AddOn}, description={Eine optionale Erweiterung für den Firefox Webbrowser}}
\newglossaryentry{gpg}{type=main, name={GnuPG}, description={Ein freies Kryptographiesystem das zum Ver- und Entschlüsseln von Daten sowie zum Erzeugen und
Prüfen digitaler Signaturen dient}}
\newglossaryentry{foss}{type=main, name={FOSS}, description={Ein Akronym für Freie Software und Open-Source-Software}}
\newglossaryentry{ascii}{type=main, name={ASCII}, description={Der American Standard Code for Information Interchange ist eine 7-Bit-Zeichenkodierung}}
\newglossaryentry{ip}{type=main, name={IP}, description={Das Internet Protocol (IP) st ein Netzwerkprotokoll und stellt die Grundlage des Internets dar}}
\newglossaryentry{dns}{type=main, name={DNS}, description={Das Domain Name System (DNS) dient zur Auflösung von Hostnamen in IP-Adressen}}
\newglossaryentry{oid}{type=main, name={OID}, description={Ein weltweit eindeutiger Bezeichner, der benutzt wird um ein Informationsobjekt zu benennen}}
