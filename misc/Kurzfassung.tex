% vim: set tw=160:

% 2010 Peter A. Menzel

\noindent
\begin{minipage}{\textwidth}
%%% minipage um Kapitel Kurzfassung UND Kapitel Abstract
%%% auf eine Seite zu binden
\chapter*{Kurzfassung}
Die zunehmende Verbreitung von Websites, auf denen es möglich ist, auch als Benutzer Inhalte zu veröffentlichen, macht es immer wichtiger, diese Inhalte auch
eindeutig, über die Grenzen einer einzelnen \glstext*{website} hinweg, einem Autor eindeutig zuordnen zu können. Um diese Zuordnung zuverlässig herstellen zu
können, bieten sich die drei kryptographischen Eigenschaften Integrität, Authentizität und Verbindlichkeit in der Form von digitalen Signaturen an. Dazu
betrachtet diese Arbeit Anforderungen, die an digitale Signaturen in einer \glstext*{website} gestellt werden. Es werden Ansätze zur Verbindung von
\gls{website} und digitaler Signatur erläutert und die Verfahren \gls{openpgp} und \gls{x509} auf ihre Tauglichkeit für diesen Einsatzzweck gegenübergestellt.
Als Resultat werden aus diesen Informationen mehrere Anforderungen für ein, noch zu entwickelndes, \glstext{firefox-addon} für \glstext*{xml-dsig} abgeleitet.

\chapter*{Abstract}
With the proliferation of websites that offer their users to publish their own content, it is becoming more and more important, that content can be assigned an
author even across the boundaries of a single website. To establish such a reliable mapping it is crucial to be able to apply the three cryptographic
properties of integrity, authenticity and non-repudiability in the form of digital signatures to any content of a website. This work researches requirements
imposed by digital signatures in a website. It identifies approaches to embedding signatures within a website. For this purpose two digital signature systems,
\glstext*{openpgp} and \glstext*{x509}, are evaluated on their suitability in the scope of a website scenario. As a result, this information is used to derive
requirements for a, yet to be developed, Firefox add-on that incorporates \glstext*{xml-dsig} to validate digital signatures in \glsplural*{website}.

\end{minipage}
