% vim: set tw=160:

% 2010 Peter A. Menzel

\noindent
\begin{minipage}{\textwidth}
%%% minipage um Kapitel Kurzfassung UND Kapitel Abstract
%%% auf eine Seite zu binden
\chapter*{Kurzfassung}
Diese Arbeit wurde als Theoriefindung zum Thema ``Digitale Signaturen in \glstext*{xhtml}'' geschrieben und behandelt die Anforderungen, die an eine Verbindung
der beiden Technologien \glstext*{xhtml} und \glstext*{xml-dsig} gestellt werden. Es wird die Funktionsweise von \glstext*{xml-dsig} beschrieben, und die damit
verwendbaren, bereits im Internet verbreiteten Verfahren zu Erstellung von digitalen Signaturen sowie zur Verteilung von Zertifikaten beziehungsweise
öffentlichen, asymmetrischen Schlüsseln, \glstext*{openpgp} und \glstext*{x509}, erläutert. Für die Verbindung zwischen \glstext*{xml-dsig} und
\glstext*{xhtml} werden mehrere Ansätze vorgestellt, die das Vorhandensein einer digitalen Signatur auf der Ebene von \glstext*{http}/\glstext*{https} oder
\glstext*{xhtml} anzeigen. Diese Informationen werden letztlich zu Anforderungen an ein, noch zu entwickelndes, AddOn für einen \glstext*{webbrowser}, welcher
\glstext*{openpgp} und auch \glstext*{x509} für seine kryptografische Funktionalität benutzt, zusammengefasst.

\chapter*{Abstract}
This work has been written as original research on the topic of "`Digital Signatures in \glstext*{xhtml}"' and deals with the requirements that a combination of
\glstext*{xhtml} and \glstext*{xml-dsig} imposes. It describes the workings of \glstext*{xml-dsig} and two popular methods already in use on the Internet for
creating digital signatures and distributing certificates or public asymmetric keys, \glstext*{openpgp} and \glstext*{x509}. Several approaches towards feasible
connections between \glstext*{xml-dsig} and \glstext*{xhtml} are introduced. They all aim to indicate the presence of a digital signature on the level of
\glstext*{http}/\glstext*{https} or \glstext*{xhtml}. \glstext*{openpgp} and \glstext*{x509} are wieghted against each other in busines and private scenarios of
in terms of key distribution strategies. This information ultimately leads to requirements for a, still to be developed, add-on for a web browser, faciliating
both \glstext*{openpgp} and \glstext*{x509} for its cryptographic functions through an abstract interface.

\end{minipage}
